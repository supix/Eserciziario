\exercise{Operatore di Confronto}
Implementare il metodo \cod{equal()} della classe \cod{AlberoBinario} (\cfr{Ex:Albero Binario}). Tale operatore viene invocato come mostrato nell'esempio di seguito:
\begin{codequote}
	boolean sonoUguali = a.equals(obj);
\end{codequote}  
dove \cod{a} � l'istanza della classe \cod{AlberoBinario} sulla quale avviene l'invocazione. Si noti che \cod{obj} � un'istanza della classe base \cod{Object} (da cui ereditano tutte le classi Java), coerentemente con l'implementazione del metodo \cod{equals()} gi� presente nella classe base, di cui questa rappresenta di fatto un \emph{override}.

Di seguito si riporta la specifica del metodo da realizzare.
  
\begin{methodslist}

\method{equals}{AlberoBinario, Object}{bool} {
� l'operatore di confronto tra un albero ed un generico oggetto. Fornisce \cod{true} se l'oggetto in ingresso � un'istanza della classe \cod{AlberoBinario} uguale (anche strutturalmente) all'albero che riceve la chiamata, \cod{false} altrimenti.
}
\end{methodslist}