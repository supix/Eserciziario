\thesolution{Operatore di Confronto}
Il metodo \cod{equals()} deve innanzitutto verificare che l'istanza di oggetto passata in ingresso sia della classe \cod{AlberoBinario}. A questo scopo si utilizza l'operatore del linguaggio \cod{instanceof}, che restituisce un valore booleano. Se questo operatore restituisce il valore \cod{false}, il metodo termina prematuramente con il valore \cod{false}. Altrimenti si procede alla verifica sull'uguaglianza strutturale tra i due alberi da confrontare.

Si noti come la variabile \cod{alb} del metodo \cod{equals()} rappresenti un ulteriore riferimento all'oggetto \cod{obj}: bench� i due riferimenti siano identici (puntino cio� alla medesima locazione di memoria), il loro tipo � differente e \cod{alb} pu� essere utilizzato come parametro attuale nella chiamata del metodo ricorsivo \cod{\_equals()}.

\inputprogram{Esercizi/OperatoreConfronto/OperatoreConfronto.java}