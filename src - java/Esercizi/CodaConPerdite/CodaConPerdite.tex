\exercise{Coda con Perdite}
Si realizzi in linguaggio Java il tipo di dato astratto \cod{Coda} mediante uso del costrutto \cod{class} del linguaggio. Il tipo degli elementi della coda sia il tipo \cod{int}.

Di seguito � riportata la specifica dei metodi pubblici da implementare per la classe \cod{Coda}.

\begin{methodslist}

\method{Coda}{int}{\emptyset} {
Costruttore con parametro intero. Il parametro indica il numero massimo di posti in coda, oltre il quale non deve essere possibile inserire ulteriori elementi.
}

\method{push}{int}{boolean} {
Accoda l'elemento specificato. Restituisce \cod{true} in caso di elemento accodato, \cod{false} altrimenti.
}

\method{top}{\emptyset}{int} {
Restituisce l'elemento di testa corrente della coda (ma non lo estrae). In caso di coda vuota il comportamento di questo metodo � non specificato.
}

\method{pop}{\emptyset}{int} {
Estrae e restituisce l'elemento di testa corrente presente in coda. In caso di coda vuota il comportamento di questo metodo � non specificato.
}

\method{pop}{int}{int} {
Estrae tanti elementi quanti specificati dal parametro di ingresso e restituisce solo il primo di questi, cio� l'elemento presente in testa precedentemente alla chiamata al metodo. Rappresenta una versione \emph{overloaded} del metodo precedente. Nel caso in cui la coda risulti vuota all'atto della chiamata al metodo, il comportamento risultante � non specificato.
}

\method{svuota}{\emptyset}{\emptyset} {
Svuota la coda.
}

\method{count}{\emptyset}{int} {
Restituisce il numero di elementi presenti nella coda.
}

\method{empty}{\emptyset}{boolean} {
Predicato vero se la coda � vuota, falso altrimenti.
}

\end{methodslist}

Si realizzi una classe \cod{Main} dotata di un metodo \cod{main()} che permetta di effettuare il collaudo della struttura dati realizzata.

Nessuno dei metodi della classe pu� operare con i canali di input/output (per es. \cod{System.out}). L'interfacciamento con l'utente per la lettura e la visualizzazione dei dati sono concessi esclusivamente all'interno della classe \cod{Main}.