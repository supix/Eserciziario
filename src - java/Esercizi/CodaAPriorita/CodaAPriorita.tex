\exercise{Coda a Priorit�}
Si realizzi in linguaggio Java il tipo di dato astratto \cod{PriorityQueue} mediante uso del costrutto \cod{class} del linguaggio. Il tipo degli elementi della coda sia il tipo \cod{int}. La struttura permette di accodare elementi che possono avere due differenti livelli di priorit�: \textsf{high} (alta) e \textsf{low} (bassa). Un elemento a bassa priorit� viene sempre accodato alla struttura. Un elemento a priorit� alta ha invece la precedenza sugli elementi a priorit� bassa, ma non sugli elementi a priorit� alta eventualmente gi� presenti nella struttura.

Di seguito � riportata la specifica dei metodi pubblici da implementare per la classe \cod{Coda}.

\begin{methodslist}

\method{PriorityQueue}{\emptyset}{\emptyset} {
Costruttore.
}

\method{pushLow}{int}{\emptyset} {
Accoda un elemento a bassa priorit�.
}

\method{pushHigh}{int}{\emptyset} {
Accoda un elemento ad alta priorit�.
}

\method{pop}{\emptyset}{int} {
Estrae e restituisce il primo elemento ad alt� priorit� o, in sua assenza, il primo elemento a bassa priorit�. In caso di coda vuota il comportamento di questo metodo � non specificato.
}

\method{svuota}{\emptyset}{\emptyset} {
Svuota la coda.
}

\method{empty}{\emptyset}{boolean} {
Predicato vero se la coda � vuota, falso altrimenti.
}

\end{methodslist}

Si realizzi una classe \cod{Main} dotata di un metodo \cod{main()} che permetta di effettuare il collaudo della struttura dati realizzata.

Nessuno dei metodi della classe pu� operare con i canali di input/output (per es. \cod{System.out}). L'interfacciamento con l'utente per la lettura e la visualizzazione dei dati sono concessi esclusivamente all'interno della classe \cod{Main}.