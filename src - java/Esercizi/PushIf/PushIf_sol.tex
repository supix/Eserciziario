\thesolution{Push If}
Nella parte privata della classe sono dichiarati i seguenti membri:

\begin{codequote}
	public class Pila {
		...
		private Record top;
		private int numEl;
		private final int maxPush;
		private int currPush;
		...
	}
\end{codequote}

La variabile membro \cod{maxpush} tiene memoria di qual � il numero di inserimenti massimi consecutivi ammessi; il suo valore � inizializzato dal costruttore al valore del parametro di ingresso e mai pi� variato durante il ciclo di vita delle istanze della classe. La sua marcatura con l'attributo \cod{final} abilita proprio tali controlli da parte del compilatore. La variabile membro \cod{currpush} tiene memoria del numero di inserimenti consecutivi correntemente effettuati. Ogni chiamata al metodo \cod{push()} deve verificare che questo parametro non ecceda il valore massimo consentito. Il metodo privato \cod{\_push()} � implementato come una classica push su coda, cos� come disponibile nell'esercizio �\ref{Ex:Push Greater}.

Di seguito si riporta l'implementazione dei metodi richiesti dalla traccia.

\inputprogram{Esercizi/PushIf/Pila.java}