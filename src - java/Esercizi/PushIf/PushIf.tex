\exercise{Push If}
Si modifichi la classe \cod{Pila} dell'esercizio �\ref{Ex:Push Greater} per renderla conforme ai metodi specificati di seguito:

\begin{methodslist}

\method{Pila}{int}{\emptyset} {
Costruttore con parametro. Il parametro di ingresso indica il numero di inserimenti massimi consecutivi possibili (vedi anche specifiche del metodo \cod{push()}).
}

\method{push}{int}{boolean} {
Aggiunge sulla pila l'elemento specificato se non � stato superato il numero massimo di inserimenti consecutivi (cio� non intervallati da alcun prelievo con il metodo \cod{pop()} o da uno svuotamento completo della lista con il metodo \cod{svuota()}). Nel caso in cui tale numero, specificato dal parametro di ingresso del costruttore, sia stato superato, l'inserimento non avviene ed il metodo restituisce \cod{false}. Altrimenti restituisce \cod{true}.
}

\method{pop}{\emptyset}{int} {
Estrae e restituisce l'elemento di testa corrente della pila. Azzera il conteggio degli inserimenti. In caso di pila vuota il comportamento di questo metodo � non specificato.
}

\method{svuota}{\emptyset}{\emptyset} {
Svuota la pila ed azzera il conteggio degli inserimenti.
}

\end{methodslist}