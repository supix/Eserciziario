\exercise{Contenitore}
Si realizzi in linguaggio Java il tipo di dato astratto \cod{Contenitore} mediante uso del costrutto \cod{class} del linguaggio. Un Contenitore pu� contenere istanze del tipo Oggetto, modellato con un nome (di classe \cod{String}) ed un peso (di tipo \cod{float}).

Inoltre, ogni contenitore pu� ospitare oggetti fino al raggiungimento di un peso complessivo massimo, oltre il quale nessun altro oggetto pu� essere ospitato.

Di seguito � riportata la specifica dei metodi pubblici da implementare per la classe \cod{Contenitore}.

\begin{methodslist}

\method{Contenitore}{float}{\emptyset}{
Costruttore con parametro di tipo \cod{float}. Il parametro indica il peso massimo raggiungibile dalla totalit� degli oggetti presenti nel contenitore.
}

\method{inserisci}{String, float}{boolean}{
Inserisce nel contenitore un oggetto avente il nome e il peso specificato. Il metodo restituisce \cod{true} se l'inserimento va a buon fine, cio� se il peso dell'elemento da inserire non eccede la capacit� residua del contenitore, \cod{false} altrimenti.
}

\method{svuota}{\emptyset}{\emptyset}{
Svuota il contenitori di tutti gli oggetti presenti in esso.
}

\method{pesoComplessivo}{\emptyset}{float}{
Restituisce il peso complessivo raggiunto dal contenitore.
}

\method{pesoResiduo}{\emptyset}{float}{
Restituisce il peso residuo per il raggiungimento della capacit� massima del contenitore.
}

\method{numElem}{\emptyset}{int}{
Restituisce il numero di oggetti presenti nel contenitore.
}

\method{toString}{\emptyset}{String}{
Stampa le coppie (nome, peso) di tutti gli oggetti presenti nel contenitore.
}

\end{methodslist}

Si realizzi una classe \cod{Main} dotata di un metodo \cod{main()} che permetta di effettuare il collaudo della struttura dati realizzata.

Nessuno dei metodi della classe pu� operare con i canali di input/output (per es. \cod{System.out}). L'interfacciamento con l'utente per la lettura e la visualizzazione dei dati sono concessi esclusivamente all'interno della classe \cod{Main}.