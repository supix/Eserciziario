\exercise{Lista Statica}
Si realizzi la struttura dati \cod{Lista} secondo un approccio all'allocazione della memoria di tipo statico. Il tipo degli elementi contenuti sia uguale al tipo \cod{int} del linguaggio. La lista sia dotata dei metodi riportati di seguito.

\begin{methodslist}

\method{Lista}{\emptyset}{\emptyset}{
Costruttore.
}

\method{inserisciInCoda}{int}{\emptyset}{
Inserisce un elemento in coda alla lista.
}

\method{svuota}{\emptyset}{\emptyset}{
Svuota la lista.
}

\method{count}{\emptyset}{int}{
Restituisce il numero degli elementi contenuti nella lista.
}

\method{toString}{\emptyset}{String}{
Restituisce una stringa contenente tutti gli elementi della lista.
}

\end{methodslist}

Nessun metodo della classe \cod{Lista} pu� utilizzare le funzionalit� di stampa (\cod{System.out}).