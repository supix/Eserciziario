\exercise{Classifica}
Si realizzi in linguaggio Java il tipo di dato astratto \cod{Classifica} mediante uso del costrutto \cod{class} del linguaggio. L'implementazione deve essere realizzata mediante puntatori ed allocazione dinamica della memoria. Gli elementi della lista siano della classe \cod{Squadra}, modellata con un nome (di classe \cod{String}) ed un punteggio (di tipo \cod{int}).

Di seguito � riportata la specifica dei metodi pubblici da implementare per la classe \cod{Classifica}.

\begin{methodslist}

\method{Classifica}{\emptyset}{\emptyset} {
Costruttore.
}

\method{aggiungi}{Nome, int}{int} {
Se la squadra non � gi� presente, la aggiunge alla struttura e le assegna il punteggio specificato. Nel caso di squadra gi� presente, aggiunge il punteggio specificato a quello gi� totalizzato. Restituisce il numero di punti correntemente totalizzati dalla squadra.
}

\method{svuota}{\emptyset}{\emptyset} {
Svuota la struttura.
}

\method{toString}{\emptyset}{String} {
Restituisce la rappresentazione testuale della struttura dati, contenente nome e punteggio di tutte le squadre in ordine decrescente di punteggio.
}

\method{count}{\emptyset}{int} {
Conta gli elementi contenuti nella struttura.
}

\end{methodslist}

Si realizzi una classe \cod{Main} dotata di un metodo \cod{main()} che permetta di effettuare il collaudo della struttura dati realizzata.

Nessuno dei metodi della classe pu� operare con i canali di input/output (per es. \cod{System.out}). L'interfacciamento con l'utente per la lettura e la visualizzazione dei dati sono concessi esclusivamente all'interno della classe \cod{Main}.

\begin{footnotesize}
Suggerimento: l'aggiornamento di un punteggio nella struttura pu� essere convenientemente realizzato attraverso la composizione di un'eliminazione ed un inserimento ordinato.
\end{footnotesize}