\exercise{Cifratore}
Implementare la classe \cod{Cifratore} con la capacit� di cifrare stringhe di caratteri attraverso uno slittamento del codice ASCII dei caratteri componenti la stringa (c.d.~codice di Cesare). L'interfaccia della classe sia la seguente.
  
\begin{methodslist}

\method{Cifratore}{int}{\emptyset} {
Costruttore della classe. Imposta la costante intera di slittamento che il cifratore utilizza per crittografare le stringhe.
}

\method{cifra}{String}{String} {
Metodo di cifratura. Accetta la stringa da cifrare e ne restituisce la versione cifrata. La cifratura consiste in uno slittamento (\texttt{shift}) dei codici ASCII di ogni singolo carattere della stringa.
}

\method{decifra}{String}{String} {
Metodo di decifratura. Accetta la stringa cifrata attraverso il metodo \cod{Cifra()} e ne restituisce nuovamente la versione decifrata.
}

\end{methodslist}