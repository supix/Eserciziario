\thesolution{Conta Nodi Sottoalbero}
Il problema posto pu� essere scomposto in due sottoproblemi:
\begin{itemize}
  \item individuare la radice del sottoalbero di cui contare i nodi;
  \item contare i nodi del sottoalbero individuato.
\end{itemize}
Solo la prima delle due operazioni suddette dipende da quale dei due metodi viene invocato, a differenza della seconda che resta inalterata. Questa considerazione suggerisce di aggiungere alla classe \cod{AlberoBinario} i seguenti metodi:

\begin{codequote}
	private int _contaNodi(Nodo n) {
		...
	}
	
	private Nodo _cercaOccorrenzaMin(Nodo n, int el) {
		...
	}

	private Nodo _cercaOccorrenzaMax(Nodo n, int el) {
		...
	}

	public int contaNodiSottoalb_Min(int el) {
		...
	}

	public int contaNodiSottoalb_Max(int el) {
		...
	}
\end{codequote}

Il metodo \cod{\_contaNodi()} restituisce il numero di nodi di un sottoalbero di cui sia fornita la radice. Il metodo \cod{\_cercaOccorrenzaMin()} restituisce il puntatore al nodo avente valore specificato e posizionato pi� in alto (livello minimo) all'interno di un albero di cui si fornisce la radice. Analogo comportamento ha il metodo \cod{\_cercaOccorrenzaMax()}. I due metodi pubblici svolgono le operazioni richieste basandosi sui metodi privati mostrati. \cod{\_cercaOccorrenzaMin()}, ad esempio, invoca il metodo ricorsivo \cod{\_cerca\-Oc\-cor\-ren\-za\-Min()} perch� individui la radice del sottoalbero; su tale radice invoca poi il metodo \cod{\_contaNodi()}.

Di seguito si riportano le implementazioni dei cinque metodi dichiarati.

\inputprogram{Esercizi/ContaNodiSottoalbero/ContaNodiSottoalbero.java}