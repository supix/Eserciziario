\exercise{Conta Nodi Sottoalbero}
Dotare la classe \cod{AlberoBinario} (\cfr{Ex:Albero Binario}) dei metodi aventi l'interfaccia specificata di seguito.

\begin{methodslist}

\method{contaNodiSottoalb\_Min}{int}{int} {
Conta i nodi del sottoalbero avente come radice l'elemento il cui valore � pari al valore del parametro di ingresso. Nel caso di occorrenze multiple, la radice viene individuata nell'elemento posizionato al livello dell'albero minore (pi� in alto) rispetto a tutti gli altri. In caso di assenza dell'elemento, il metodo restituisce \cod{null}. Si consideri anche la radice del sottoalbero nel conteggio degli elementi.
}

\method{contaNodiSottoalb\_Max}{int}{int} {
Conta i nodi del sottoalbero avente come radice l'elemento il cui valore � pari al valore del parametro di ingresso. Nel caso di occorrenze multiple, la radice viene individuata nell'elemento posizionato al livello dell'albero maggiore (pi� in basso) rispetto a tutti gli altri. In caso di assenza dell'elemento, il metodo restituisce \cod{null}. Si consideri anche la radice del sottoalbero nel conteggio degli elementi.
}

\end{methodslist}