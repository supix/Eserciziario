\exercise{Agenzia Matrimoniale}
Si realizzi in linguaggio Java il tipo di dato astratto \cod{AgenziaMatrimoniale} mediante uso del costrutto \cod{class} del linguaggio. L'implementazione deve essere realizzata mediante puntatori ed allocazione dinamica della memoria. Gli elementi della lista siano di tipo \cod{Persona} modellata con il nome (di classe \cod{String}), il sesso (di tipo \cod{boolean}) e il coniuge (di classe \cod{Persona}).

Di seguito � riportata la specifica dei metodi pubblici da implementare per la classe \cod{AgenziaMatrimoniale}.

\begin{methodslist}

\method{AgenziaMatrimoniale}{\emptyset}{\emptyset} {
Costruttore.
}

\method{aggiungiPersona}{String, boolean}{boolean} {
Aggiunge alla struttura la persona avente nome specificato attraverso i parametri di ingresso, e indica se � maschio (parametro di ingresso booleano pari a \cod{true}) o femmina (parametro di ingresso booleano pari a \cod{false}). Restituisce \cod{true} in caso di inserimento avvenuto, \cod{false} altrimenti (se esiste gi� una persona con lo stesso nome).
}

\method{sposa}{String, String}{boolean} {
Marca come sposate le due persone presenti nella struttura ed aventi nomi specificati dai parametri di ingresso. Restituisce l'esito dell'operazione. L'operazione fallisce nei casi seguenti:
\begin{itemize}
\item uno o entrambi i nomi non sono presenti nella lista;
\item le persone specificate sono dello stesso sesso;
\item una o entrambe le persone risultano gi� sposate.
\end{itemize}
}

\method{coniugato}{String}{boolean, boolean} {
Restituisce due valori booleani. Il primo indica se il nome specificato � presente o meno nella lista. Se tale valore � vero, il secondo valore restituito � pari a vero se la persona dal nome specificato � coniugata, falso altrimenti.
}

\method{numeroSposi}{\emptyset}{int} {
Restituisce il numero delle persone coniugate nella struttura.
}

\method{numeroCoppie}{\emptyset}{int} {
Restituisce il numero di coppie di sposi presenti nella struttura.
}

\method{svuota}{\emptyset}{\emptyset} {
Svuota la struttura.
}

\method{toString}{\emptyset}{String} {
Restituisce la rappresentazione testuale della struttura.
}

\end{methodslist}

Si realizzi una classe \cod{Main} dotata di un metodo \cod{main()} che permetta di effettuare il collaudo della struttura dati realizzata.

Nessuno dei metodi della classe pu� operare con i canali di input/output (per es. \cod{System.out}). L'interfacciamento con l'utente per la lettura e la visualizzazione dei dati sono concessi esclusivamente all'interno della classe \cod{Main}.