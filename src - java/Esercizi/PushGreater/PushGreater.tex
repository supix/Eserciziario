\exercise{Push Greater}
Si realizzi in linguaggio Java il tipo di dato astratto \cod{Pila} mediante uso del costrutto \cod{class} del linguaggio e ricorrendo ad un'implementazione dinamica. Il tipo degli elementi della pila sia il tipo \cod{int}.

Di seguito � riportata la specifica dei metodi pubblici da implementare per la classe \cod{Pila}.

\begin{methodslist}

\method{Pila}{\emptyset}{\emptyset} {
Costruttore senza parametri.
}

\method{push}{int}{\emptyset} {
Aggiunge sulla pila l'elemento specificato.
}

\method{pushGreater}{int}{boolean} {
Aggiunge sulla pila l'elemento specificato esclusivamente se esso � maggiore dell'elemento di testa corrente. Nel caso in cui la pila sia vuota l'aggiunta � sempre eseguita. Restituisce \cod{true} oppure \cod{false} a seconda che l'aggiunta sia stata eseguita oppure no.
}

\method{top}{\emptyset}{int} {
Restituisce l'elemento di testa corrente della pila (ma non lo estrae). In caso di pila vuota il comportamento di questo metodo � non specificato.
}

\method{pop}{\emptyset}{int} {
Estrae e restituisce l'elemento di testa corrente della pila. In caso di pila vuota il comportamento di questo metodo � non specificato.
}

\method{svuota}{\emptyset}{\emptyset} {
Svuota la pila.
}

\method{count}{\emptyset}{int} {
Restituisce il numero di elementi presenti nella pila.
}

\method{empty}{\emptyset}{boolean} {
Predicato vero se la pila � vuota, falso altrimenti.
}
\end{methodslist}

Si realizzi una classe \cod{Main} dotata di un metodo \cod{main()} che permetta di effettuare il collaudo della struttura dati realizzata.

Nessuno dei metodi della classe pu� operare con i canali di input/output (per es. \cod{System.out}). L'interfacciamento con l'utente per la lettura e la visualizzazione dei dati sono concessi esclusivamente all'interno della classe \cod{Main}.