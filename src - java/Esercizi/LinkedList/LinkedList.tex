\exercise{Lista Semplicemente Collegata}
Si realizzi la struttura dati \cod{Lista} secondo l'implementazione della lista semplicemente collegata (ogni elemento punta al successivo). Il tipo degli elementi contenuti sia uguale al tipo \cod{int} del linguaggio. La lista sia dotata dei metodi riportati di seguito.

\begin{methodslist}
\method{Lista}{\emptyset}{\emptyset} {
 Costruttore senza parametri.
}

\method{Lista}{Lista}{\emptyset} {
  Costruttore di copia.
}

\method{inserisci}{int}{\emptyset} {
  Inserimento in testa alla lista.
}

\method{numeroElementi}{\emptyset}{int} {
  Restituisce il numero degli elementi contenuti nella lista.
}

\method{svuota}{\emptyset}{\emptyset} {
  Svuota la lista.
}

\method{elimina}{int}{\emptyset} {
  Elimina un elemento dalla lista, se presente.
}

\method{toString}{\emptyset}{String} {
  Restituisce in forma testuale la rappresentazione della struttura costituita da tutti gli elementi contenuti nella lista.
}

\method{ricerca}{int}{boolean} {
  Predicato indicante la presenta di un elemento.
}

Si realizzi una classe \cod{Main} dotata del metodo \cod{public static void main()} che permetta di effettuare il collaudo della struttura dati realizzata.

\end{methodslist}