\exercise{Lista Della Spesa}
Si realizzi in linguaggio Java il tipo di dato astratto \cod{ListaDellaSpesa} mediante uso del costrutto \cod{class} del linguaggio e ricorrendo ad un'implementazione dinamica. Gli elementi della lista siano del tipo \cod{Articolo}, che contiene gli attributi nome e quantit�, di tipo \cod{String} e \cod{float}, rispettivamente.

Di seguito si riporta la specifica dei metodi da implementare.

\begin{methodslist}

\method{ListaDellaSpesa}{\emptyset}{\emptyset} {
Costruttore.
}

\method{aggiungi}{String, float}{float} {
Aggiunge alla lista la quantit� specificata in corrispondenza dell'articolo indicato. Il metodo restituisce la quantit� con cui l'articolo specificato � presente nella lista in seguito all'aggiunta.
}

\method{elimina}{String}{boolean} {
Elimina dalla lista l'elemento avente il nome specificato (se presente). Il metodo restituisce \cod{true} se � stato cancellato un elemento, \cod{false} altrimenti.
}

\method{getQuantita}{String}{float} {
Restituisce la quantit� dell'elemento presente nella lista ed avente il nome specificato. Se l'elemento non � presente restituisce zero.
}

\method{svuota}{\emptyset}{\emptyset} {
Svuota la lista della spesa.
}

\method{toString}{\emptyset}{String} {
Converte in una stringa il contenuto dell'intera lista nel formato \cod{Nome1: Quantit�1, Nome2: Quantit�2, \ldots}.

\begin{footnotesize}
Suggerimento: si utilizzi la classe standard \cod{StringBuilder} per la costruzione incrementale della stringa da restituire.
\end{footnotesize}
}

\end{methodslist}

Si realizzi una classe \cod{Main} dotata di un metodo \cod{main()} che permetta di effettuare il collaudo della struttura dati realizzata.

Nessuno dei metodi della classe pu� operare con i canali di input/output (per es. \cod{System.out}). L'interfacciamento con l'utente per la lettura e la visualizzazione dei dati sono concessi esclusivamente all'interno della classe \cod{Main}.