\thesolution{Cerca Foglia}
I metodi \cod{cercaFoglia()} e \cod{cercaNodo()} devono restituire un'informazione strutturata, costituita da due valori logici. Come di consueto, saranno utilizzate allo scopo delle classi appositamente definite. In questo caso si � scelto di realizzare questa classi come \emph{classi immutabili}. Ci� significa che il valore di ogni loro istanza viene definito all'atto della creazione e non pu� essere variato per tutto il ciclo di vita. Per esempio, il costruttore della classe \cod{RisultatoRicercaFoglia} � definito come segue:

\enablelstnum
\begin{codequote}
		public RisultatoRicercaFoglia(boolean trovato,
		        boolean foglia) {
			this.trovato = trovato;
			this.foglia = foglia;
		}
\end{codequote}
\disablelstnum

Le righe $1$ e $2$ contengono i parametri di ingresso, utili ad assegnare il valore delle omonime variabili private contenute nella classe. L'omonimia non crea alcun problema poich�, nello scope in cui sono visibili sia le variabili private che i parametri locali (righe $3$ e $4$), viene risolta utilizzando il prefisso \cod{this}. Laddove � presente \cod{this} ci si sta riferendo alla variabile membro privata. Laddove \cod{this} non � presente, si fa invece riferimento al parametro locale.

Di seguito � riportata l'implementazione dei metodi pubblici e privati utili a rispondere alla traccia.

\inputprogram{Esercizi/CercaFoglia/CercaFoglia.java}