\thesolution{Numero Elementi}
La tecnica pi� semplice per effettuare il conteggio del numero di elementi contenuti in un albero, consiste nel definire un membro privato di tipo intero non negativo atto a memorizzare tale valore. Il valore del membro viene alterato da tutti i metodi della struttura che modificano il numero di nodi presenti in essa (inserimento, eliminazione, svuotamento, ecc.).

Qui si mostrer� un approccio differente, in generale meno efficiente, consistente in un metodo ricorsivo che calcola il numero di elementi mediante una visita completa dell'albero.

\bigskip
\inputprogram{Esercizi/NumeroElementi/NumElem.java}