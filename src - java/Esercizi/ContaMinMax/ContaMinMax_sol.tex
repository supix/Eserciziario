\thesolution{Conta Min e Max}
Il conteggio degli elementi compresi entro un certo intervallo pu� essere svolto mediante una visita dell'albero. Data la propriet� di ordinamento dell'albero, non � peraltro necessario visitare completamente la struttura. Si consideri per esempio il caso in cui si debbano conteggiare gli elementi compresi nell'intervallo $[10,20]$. In occasione della visita di un ipotetico elemento pari a $5$, � inutile procedere verso il sottoalbero sinistro di tale elemento, che non ha possibilit� di fornire un contributo al conteggio in corso.

\inputprogram{Esercizi/ContaMinMax/ContaMinMax.java}