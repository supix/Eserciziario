\exercise{Coda}
Si realizzi in linguaggio Java il tipo di dato astratto \cod{Coda} mediante uso del costrutto \cod{class} del linguaggio e ricorrendo ad un'implementazione dinamica. Il tipo degli elementi della coda sia il tipo \cod{int}.

Di seguito � riportata la specifica dei metodi pubblici da implementare per la classe \cod{Coda}.

\begin{methodslist}

\method{Coda}{\emptyset}{\emptyset} {
Costruttore senza parametri.
}

\method{push}{int}{\emptyset} {
Accoda l'elemento specificato.
}

\method{top}{\emptyset}{int} {
Restituisce l'elemento di testa corrente della coda (ma non lo estrae). In caso di coda vuota il comportamento di questo metodo � non specificato.
}

\method{pop}{\emptyset}{int} {
Estrae e restituisce l'elemento di testa corrente presente in coda. In caso di coda vuota il comportamento di questo metodo � non specificato.
}

\method{somma}{\emptyset}{int} {
Restituisce la somma di tutti gli elementi presenti in coda.
}

\method{svuota}{\emptyset}{\emptyset} {
Svuota la coda.
}

\method{count}{\emptyset}{int} {
Restituisce il numero di elementi presenti nella coda.
}

\method{empty}{\emptyset}{boolean} {
Predicato vero se la coda � vuota, falso altrimenti.
}

\end{methodslist}

Si realizzi una classe \cod{Main} dotata di un metodo \cod{main()} che permetta di effettuare il collaudo della struttura dati realizzata.

Nessuno dei metodi della classe pu� operare con i canali di input/output (per es. \cod{System.out}). L'interfacciamento con l'utente per la lettura e la visualizzazione dei dati sono concessi esclusivamente all'interno della classe \cod{Main}.