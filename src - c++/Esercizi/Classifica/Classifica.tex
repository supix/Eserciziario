\exercise{Classifica}
Si realizzi in linguaggio C++ il tipo di dato astratto \cod{Classifica} mediante uso del costrutto \cod{class} del linguaggio. L'implementazione deve essere realizzata mediante puntatori ed allocazione dinamica della memoria. Gli elementi della lista siano di tipo \cod{TElem}, definito nel modo seguente:

\begin{codequote}
  const int NMAX = 50;
  typedef char Nome[NMAX]; //Nome delle squadre

  typedef struct {
    Nome n;
    unsigned int punteggio;
  } Squadra;

  typedef Squadra TElem;
\end{codequote}

Di seguito � riportata la specifica dei metodi pubblici da implementare per la classe \cod{Classifica}.

\begin{methodslist}

\method{Classifica}{\emptyset}{\emptyset} {
Costruttore.
}

\method{\~{}Classifica}{\emptyset}{\emptyset} {
Distruttore.
}

\method{Aggiungi}{Nome,unsigned int}{unsigned int} {
Se la squadra non � gi� presente, la aggiunge alla struttura e le assegna il punteggio specificato. Nel caso di squadra gi� presente, aggiunge il punteggio specificato a quello gi� totalizzato. Restituisce il numero di punti correntemente totalizzati dalla squadra.
}

\method{Svuota}{\emptyset}{\emptyset} {
Svuota la struttura.
}

\method{Stampa}{\emptyset}{\emptyset} {
Stampa la classifica delle squadre presenti nella struttura, in ordine decrescente di punteggio.
}

\method{Count}{\emptyset}{unsigned int} {
Conta gli elementi contenuti nella struttura.
}

\end{methodslist}

L'unico metodo della classe \cod{Classifica} che pu� utilizzare lo standard-output (\cod{cout}) � il metodo \cod{Stampa()}. Gli altri metodi (pubblici, privati o protetti) non possono fare uso degli oggetti per l'I/O.

Si realizzi una funzione \cod{main()} che permetta di effettuare il collaudo della struttura dati realizzata.

Suggerimento: l'aggiornamento di un punteggio nella struttura pu� essere convenientemente realizzato attraverso la composizione di un'eliminazione ed un inserimento ordinato.