\exercise{Coda a Priorit�}
Si realizzi in linguaggio C++ il tipo di dato astratto \cod{PriorityQueue} mediante uso del costrutto \cod{class} del linguaggio. Il tipo \cod{TElem} degli elementi della coda sia il tipo \cod{int}. La struttura permette di accodare elementi che possono avere due differenti livelli di priorit�: \textsf{high} (alta) e \textsf{low} (bassa). Un elemento a bassa priorit� viene sempre accodato alla struttura. Un elemento a priorit� alta ha invece la precedenza sugli elementi a priorit� bassa, ma non sugli elementi a priorit� alta eventualmente gi� presenti nella struttura.

Di seguito � riportata la specifica dei metodi pubblici da implementare per la classe \cod{Coda}.

\begin{methodslist}

\method{PriorityQueue}{\emptyset}{\emptyset} {
Costruttore.
}

\method{\~{}PriorityQueue}{\emptyset}{\emptyset} {
Distruttore.
}

\method{PushLow}{TElem}{\emptyset} {
Accoda un elemento a bassa priorit�.
}

\method{PushHigh}{TElem}{\emptyset} {
Accoda un elemento ad alta priorit�.
}

\method{Pop}{\emptyset}{TElem} {
Estrae e restituisce il primo elemento ad alt� priorit� o, in sua assenza, il primo elemento a bassa priorit�. In caso di coda vuota il comportamento di questo metodo � non specificato.
}

\method{Svuota}{\emptyset}{\emptyset} {
Svuota la coda.
}

\method{Empty}{\emptyset}{bool} {
Predicato vero se la coda � vuota, falso altrimenti.
}

\end{methodslist}

Si realizzi una funzione \cod{main()} che permetta di effettuare il collaudo della struttura dati realizzata.

Nessuno dei metodi della classe pu� utilizzare operazioni che coinvolgono gli stream di input ed output (\cod{cin} e \cod{cout}). La scrittura e la lettura su stream sono concesse esclusivamente all'interno del programma \cod{main()}.