\exercise{Agenzia Matrimoniale}
Si realizzi in linguaggio C++ il tipo di dato astratto \cod{AgenziaMatrimoniale} mediante uso del costrutto \cod{class} del linguaggio. L'implementazione deve essere realizzata mediante puntatori ed allocazione dinamica della memoria. Gli elementi della lista siano di tipo \cod{TElem}, definito nel modo seguente:

\begin{codequote}
  const int NMAX = 50;
  typedef char Nome[NMAX]; //Nome Persona

  struct persona;
  typedef struct Persona{
    Nome n;
    bool maschio;
    Persona* coniuge;
  };
  
  typedef Persona TElem;
\end{codequote}  

Di seguito � riportata la specifica dei metodi pubblici da implementare per la classe \cod{AgenziaMatrimoniale}.

\begin{methodslist}

\method{AgenziaMatrimoniale}{\emptyset}{\emptyset} {
Costruttore.
}

\method{\~{}AgenziaMatrimoniale }{\emptyset}{\emptyset} {
Distruttore.
}

\method{AggiungiPersona}{Nome,bool}{bool} {
Aggiunge alla struttura la persona avente nome specificato attraverso i parametri di ingresso, e indica se � maschio (parametro di ingresso pari a \cod{true}) o femmina (parametro di ingresso pari a \cod{false}) Restituisce \cod{true} in caso di inserimento avvenuto, \cod{false} altrimenti (se esiste gi� una persona con lo stesso nome).
}

\method{Sposa}{Nome,Nome}{bool} {
Marca come sposate le due persone presenti nella struttura ed aventi nomi specificati dai parametri di ingresso. Restituisce l'esito dell'operazione. L'operazione fallisce nei casi seguenti:
\begin{itemize}
\item uno o entrambi i nomi non sono presenti nella lista;
\item le persone specificate sono dello stesso sesso;
\item una o entrambe le persone risultano gi� sposate.
\end{itemize}
}

\method{Coniugato}{Nome}{bool, bool} {
Restituisce due valori booleani. Il primo indica se il nome specificato � presente o meno nella lista. Se tale valore � vero, il secondo valore restituito � pari a vero se la persona dal nome specificato � coniugata, falso altrimenti.
}

\method{NumeroSposi}{\emptyset}{unsigned int} {
Restituisce il numero delle persone coniugate nella struttura.
}

\method{NumeroCoppie}{\emptyset}{unsigned int} {
Restituisce il numero di coppie di sposi presenti nella struttura.
}

\method{Svuota}{\emptyset}{\emptyset} {
Svuota la struttura.
}

\method{Stampa}{\emptyset}{\emptyset} {
Stampa il contenuto della struttura (vedi esempio ???).
}

\end{methodslist}

L'unico metodo della classe \cod{AgenziaMatrimoniale} che pu� utilizzare lo standard-output (\cod{cout}) � il metodo \cod{Stampa()}. Gli altri metodi (pubblici, privati o protetti) non possono fare uso degli oggetti per l'I/O.

Si realizzi una funzione \cod{main()} che permetta di effettuare il collaudo della struttura dati realizzata.