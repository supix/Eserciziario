\exercise{Lista Statica}
Si realizzi la struttura dati \cod{Lista} secondo un approccio all'allocazione della memoria di tipo statico. Il tipo \cod{TElem} degli elementi contenuti sia uguale al tipo \cod{int} del linguaggio. La lista sia dotata dei metodi riportati di seguito.

\begin{methodslist}

\method{Lista}{\emptyset}{\emptyset}{
Costruttore.
}

\method{\~{}Lista}{\emptyset}{\emptyset}{
Distruttore.
}

\method{InserisciInCoda}{TElem}{\emptyset}{
Inserisce un elemento in coda alla lista.
}

\method{Svuota}{\emptyset}{\emptyset}{
Svuota la lista.
}

\method{Count}{\emptyset}{int}{
Restituisce il numero degli elementi contenuti nella lista.
}

\method{Stampa}{\emptyset}{\emptyset}{
Stampa sullo standard output tutti gli elementi contenuti nella lista.
}

\end{methodslist}

L'unico metodo della classe Lista che pu� utilizzare lo standard-output (\cod{cout}) � il metodo \cod{Stampa()}. Gli altri metodi (pubblici, privati o protetti) non possono fare uso delle funzionalit� di stampa.