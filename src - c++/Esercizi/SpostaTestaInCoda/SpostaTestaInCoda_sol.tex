\thesolution{Sposta Testa in Coda}
Per svolgere l'operazione si fa uso di un metodo di supporto \cod{getPuntCoda()} deputato a restituire il puntatore all'elemento di coda della lista, se esistemte. Si noti che nessun elemento viene creato (\cod{new}) o distrutto (\cod{delete}), ma l'operazione � effettuata esclusivamente mediante ricollocazione di puntatori.

\bigskip
\inputprogram{Esercizi/SpostaTestaInCoda/SpostaTestaInCoda.cpp}