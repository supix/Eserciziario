\exercise{Coda}
Si realizzi in linguaggio C++ il tipo di dato astratto \cod{Coda} mediante uso del costrutto \cod{class} del linguaggio e ricorrendo ad un'implementazione dinamica. Il tipo \cod{TElem} degli elementi della coda sia il tipo \cod{int}.

Di seguito � riportata la specifica dei metodi pubblici da implementare per la classe \cod{Coda}.

\begin{methodslist}

\method{Coda}{\emptyset}{\emptyset} {
Costruttore senza parametri.
}

\method{\~{}Coda}{\emptyset}{\emptyset} {
Distruttore.
}

\method{Push}{TElem}{\emptyset} {
Accoda l'elemento specificato.
}

\method{Top}{\emptyset}{TElem} {
Restituisce l'elemento di testa corrente della coda (ma non lo estrae). In caso di coda vuota il comportamento di questo metodo � non specificato.
}

\method{Pop}{\emptyset}{TElem} {
Estrae e restituisce l'elemento di testa corrente presente in coda. In caso di coda vuota il comportamento di questo metodo � non specificato.
}

\method{Somma}{\emptyset}{TElem} {
Restituisce la somma di tutti gli elementi presenti in coda.
}

\method{Svuota}{\emptyset}{\emptyset} {
Svuota la coda.
}

\method{Count}{\emptyset}{unsigned int} {
Restituisce il numero di elementi presenti nella coda.
}

\method{Empty}{\emptyset}{bool} {
Predicato vero se la coda � vuota, falso altrimenti.
}

\end{methodslist}

Si realizzi una funzione \cod{main()} che permetta di effettuare il collaudo della struttura dati realizzata.

Nessuno dei metodi della classe pu� utilizzare operazioni che coinvolgono gli stream di input ed output (\cod{cin} e \cod{cout}). La scrittura e la lettura su stream sono concesse esclusivamente all'interno del programma \cod{main()}.