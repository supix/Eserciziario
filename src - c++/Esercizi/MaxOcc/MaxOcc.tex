\exercise{Occorrenza Massima}
Modificare la classe \cod{AlberoBinario} (\cfr{Ex:Albero Binario}) per prevedere un'occorrenza massima degli elementi in esso inseriti. Pi� precisamente, il costruttore deve accettare come parametro di ingresso un numero intero positivo (per es. \cod{maxocc}); l'inserimento di un nuovo elemento nell'albero deve andare a buon fine solo se tale elemento � presente con occorrenza minore di \cod{maxocc}.

Di seguito � riportata la specifica dei due metodi pubblici da implementare per la classe \cod{AlberoBinario}.

\begin{methodslist}

\method{AlberoBinario}{unsigned int}{\emptyset} {
Costruttore con parametro di ingresso di tipo intero non negativo. Il parametro di ingresso rappresenta l'occorrenza massima con cui gli elementi potranno essere presenti nell'albero.
}

\method{Inserisci}{TElem}{bool}{
Inserisce l'elemento specificato nell'albero solo se esso � presente con occorrenza minore dell'occorrenza massima specificata nel costruttore.

Il metodo restituisce \cod{true} o \cod{false} a seconda che l'inserimento sia avvenuto o meno.
}

\end{methodslist}