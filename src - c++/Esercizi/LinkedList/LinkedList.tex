\exercise{Lista Semplicemente Collegata}
Si realizzi la struttura dati \cod{Lista}. Il tipo \cod{TElem} degli elementi contenuti sia uguale al tipo \cod{int} del linguaggio. La lista sia dotata dei metodi riportati di seguito.

\begin{methodslist}
\method{Lista}{\emptyset}{\emptyset} {
 Costruttore senza parametri.
}

\method{Lista}{Lista}{\emptyset} {
  Costruttore di copia.
}

\method{\~{}Lista}{\emptyset}{\emptyset} {
  Distruttore.
}

\method{Inserisci}{TElem}{\emptyset} {
  Inserimento in testa alla lista.
}

\method{NumeroElementi}{\emptyset}{int} {
  Restituisce il numero degli elementi contenuti nella lista.
}

\method{Svuota}{\emptyset}{\emptyset} {
  Svuota la lista.
}

\method{Elimina}{TElem}{\emptyset} {
  Elimina un elemento dalla lista, se presente.
}

\method{Stampa}{\emptyset}{\emptyset} {
  Stampa sullo standard output tutti gli elementi contenuti nella lista.
}

\method{Ricerca}{TElem}{bool} {
  Predicato indicante la presenta di un elemento.
}

L'unico metodo della classe \cod{Lista} che pu� utilizzare direttamente o indirettamente lo standard-output (\cod{cout}) � il metodo \cod{Stampa()} (oltre che gli eventuali metodi ricorsivi di supporto ad esso). Gli altri metodi (pubblici, privati o protetti) non possono fare uso delle funzionalit� di stampa.

Si realizzi una funzione \cod{main()} che permetta di effettuare il collaudo della struttura dati realizzata.

\end{methodslist}