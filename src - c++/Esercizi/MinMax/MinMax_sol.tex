\thesolution{Min e Max}
La ricerca del minimo e del massimo possono essere condotte secondo un approccio iterativo. Nel listato che segue, si assume inizialmente che il minimo ed il massimo siano entrambi rappresentati dall'elemento di testa (linee 3 e 4). Successivamente si scandiscono in sequenza gli elementi della lista. Ogni volta che viene individuato un elemento minore del minimo corrente (linea 8), il minimo corrente viene aggiornato (linea 9). Analogo discorso vale per il massimo (linee 10 e 11).

\enablelstnum
\inputprogram{Esercizi/MinMax/MinMax.cpp}