\thesolution{Conta Nodi Sottoalbero}
Il problema posto pu� essere scomposto in due sottoproblemi:
\begin{itemize}
  \item individuare la radice del sottoalbero di cui contare i nodi;
  \item contare i nodi del sottoalbero individuato.
\end{itemize}
Solo la prima delle due operazioni suddette dipende da quale dei due metodi viene invocato, a differenza della seconda che resta inalterata. Questa considerazione suggerisce di aggiungere alla classe \cod{AlberoBinario} i seguenti metodi:

\begin{codequote}
  class AlberoBinario {
  private:
    ...
    
    unsigned int _ContaNodi(const PNodo& n) const;
    PNodo _CercaOccorrenzaMin(const PNodo& n,
      const TElem& el) const;
    PNodo _CercaOccorrenzaMax(const PNodo& n,
      const TElem& el) const;
   public:
    ...

    unsigned int ContaNodiSottoalb_Min(const TElem& el) const;
    unsigned int ContaNodiSottoalb_Max(const TElem& el) const;
  };
\end{codequote}

Il metodo \cod{\_ContaNodi()} restituisce il numero di nodi di un sottoalbero di cui sia fornita la radice. Il metodo \cod{\_CercaOccorrenzaMin()} restituisce il puntatore al nodo avente valore specificato e posizionato pi� in alto (livello minimo) all'interno di un albero di cui si fornisce la radice. Analogo comportamento ha il metodo \cod{\_CercaOccorrenzaMax()}. I due metodi pubblici svolgono le operazioni richieste basandosi sui metodi privati mostrati. \cod{\_CercaOccorrenzaMin()}, ad esempio, invoca il metodo ricorsivo \cod{\_Cerca\-Oc\-cor\-ren\-za\-Min()} perch� individui la radice del sottoalbero; su tale radice invoca poi il metodo \cod{\_ContaNodi()}.

Di seguito si riportano le implementazioni dei cinque metodi dichiarati.

\inputprogram{Esercizi/ContaNodiSottoalbero/ContaNodiSottoalbero.cpp}