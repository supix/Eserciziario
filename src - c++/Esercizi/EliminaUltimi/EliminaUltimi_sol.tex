\thesolution{Elimina Ultimi}
Il metodo \cod{LasciaPrimi()} richiede di eliminare gli ``elementi di coda'' della lista, preservandone i primi \cod{n}. Bisogna dapprima considerare i seguenti casi degeneri:

\begin{itemize}
\item il numero di elementi da conservare � maggiore del numero di elementi presenti nella lista: nessun elemento va eliminato (righe 2--3);
\item il numero degli elementi da conservare � pari a zero: tutti gli elementi vanno eliminati (righe 5--9).
\end{itemize}

Negli altri casi, bisogna dapprima scorrere attraverso le prime \cod{n} posizioni della lista (righe 11--16); i restanti elementi dovranno essere eliminati, operando similmente a come accade per il metodo \cod{Svuota()} (righe 26--30). L'implementazione risultante � la seguente.

\bigskip
\enablelstnum
\inputprogram{Esercizi/EliminaUltimi/LasciaPrimi.cpp}
\bigskip

Il metodo \cod{EliminaUltimi()} deve eliminare gli ultimi \cod{n} elementi. Esso non differisce nella sostanza dal precedente metodo, e pu� essere pertanto implementato nei termini di quest'ultimo.

\bigskip
\inputprogram{Esercizi/EliminaUltimi/EliminaUltimi.cpp}
\bigskip