\exercise{Votazioni}
Si supponga di voler gestire un exit-poll elettorale. Ad ogni intervistato all'uscita dal seggio si chiede il partito per cui ha votato. In ogni momento bisogna poi essere in grado di dire quanti voti ha ottenuto ciascun partito e qual � la distribuzione dei voti tra i partiti.
Mediante l'uso del costrutto \cod{class} del linguaggio C++, si realizzi una struttura dati adatta all'uopo. Si supponga, per semplicit�, che ogni partito � identificato con un codice intero, e si ignorino i voti bianchi e nulli. Di seguito � riportata la specifica dei metodi pubblici da implementare per la classe \cod{Votazioni}.

\begin{methodslist}

\method{Votazioni}{\emptyset}{\emptyset} {
Costruttore.
}

\method{\~{}Votazioni}{\emptyset}{\emptyset} {
Distruttore.
}

\method{AggiungiVoto}{unsigned int}{unsigned int} {
Aggiunge un voto al partito avente il codice specificato dal parametro di ingresso. Restituisce il numero di voti accumulati fino a quel momento dal partito.
}

\method{Svuota}{\emptyset}{\emptyset} {
Svuota la struttura.
}

\method{GetVotiPartito}{unsigned int}{unsigned int} {
Restituisce il numero di voti ottenuto dal partito avente il codice specificato dal parametro di ingresso.
}

\method{GetNumeroVoti}{\emptyset}{unsigned int} {
Restituisce il numero totale di voti.
}

\method{GetSituazione}{\emptyset}{\emptyset} {
Stampa a video un riepilogo dei voti complessivamente registrati nella struttura.
}

\end{methodslist}

L'unico metodo della classe \cod{Votazioni} che pu� utilizzare lo standard-output (\cod{cout}) � il metodo \cod{GetSituazione()}. Gli altri metodi (pubblici, privati o protetti) non possono fare uso delle funzionalit� di stampa.

Si realizzi una funzione \cod{main()} che permetta di effettuare il collaudo della struttura dati realizzata.