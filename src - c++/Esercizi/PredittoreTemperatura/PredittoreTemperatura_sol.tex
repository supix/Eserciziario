\thesolution{Predittore di Temperatura}
Il metodo \cod{EstimateTemp()} deve effettuare un'estrapolazione lineare della temperatura basandosi sui dati delle ultime due letture comunicate. La formula da utilizzare � la seguente:

%                  T2-T1
%            ET = ------- (t-t1) + T1;
%                  t2-t1

$$\hat{T} = \frac{T_2-T_1}{t_2-t_1}(t-t_1)+T_1;$$
                  
dove $\hat{T}$ � la stima della temperatura all'istante $t$; $T_1$, $T_2$, $t_1$ e $t_2$ sono le ultime due letture della temperatura ed i relativi due istanti di lettura, rispettivamente.

\bigskip
\begin{footnotesize}
N.B.: Variando l'implementazione del metodo \cod{EstimateTemp()} (ed eventualmente la sezione \cod{private} della classe) diviene possibile operare stime pi� accurate della temperatura; si potrebbe per esempio pensare di operare estrapolazioni di ordine superiore al primo. Per giunta ci�, non alterando l'interfaccia della classe, non avrebbe alcuna ripercussione negativa sui moduli utenti del predittore.
\end{footnotesize}

\bigskip
\inputprogram{Esercizi/PredittoreTemperatura/TempPredictor.cpp}