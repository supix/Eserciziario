\exercise{Parco Pattini}
La ditta Sax gestisce una pista di pattinaggio e dispone di un parco pattini. I pattini, tutti dello stesso modello, vengono fittati ai clienti a tempo, in base alla taglia della calzatura richiesta. Si implementi in linguaggio C++ la classe \cod{ParcoPattini} utile ad una prima automatizzazione nella gestione della pista. Data la definizione del tipo Taglia:

\begin{codequote}
  typedef unsigned int Taglia;
\end{codequote}

si implementi la struttura conformemente all'interfaccia specificata di seguito.

\begin{methodslist}

\method{ParcoPattini}{\emptyset}{\emptyset}{
Costruttore senza parametri. Inizializza una struttura che contiene un parco pattini vuoto.
}

\method{\~{}ParcoPattini}{\emptyset}{\emptyset}{
Distruttore.
}

\method{AggiungiPattini}{Taglia}{\emptyset}{
Aggiunge al parco un paio di pattini della misura specificata.
}

\method{Svuota}{\emptyset}{\emptyset}{
Svuota il parco pattini.
}

\method{NumeroTotPattini}{\emptyset}{unsigned int}{
Restituisce il numero di paia di pattini che costituiscono l'intero parco.
}

\method{Fitta}{Taglia}{bool}{
Marca come ``fittati'' un paio di pattini della taglia specificata dal parametro di ingresso. Il metodo restituisce \cod{true} se esiste almeno un paio di pattini della taglia specificata, \cod{false} altrimenti.

}

\method{Disponibilita}{Taglia}{unsigned int}{
Restituisce il numero di paia di pattini disponibili per la taglia specificata.
}

\method{NumeroPattini}{Taglia}{unsigned int}{
Restituisce il numero di paia di pattini appartenenti al parco, di data taglia (indipendentemente dal loro stato).
}

\method{Restituzione}{Taglia}{bool}{
Marca nuovamente come ``disponibile'' un paio di pattini della taglia specificata. Il metodo restituisce \cod{true} se effettivamente esisteva un paio di pattini della taglia specificata marcati come ``fittati'', \cod{false} altrimenti.
}

\method{Stampa}{\emptyset}{\emptyset}{
Stampa a video lo stato dell'intero parco pattini.
}

\end{methodslist}