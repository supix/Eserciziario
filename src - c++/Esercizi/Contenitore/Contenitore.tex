\exercise{Contenitore}
Si realizzi in linguaggio C++ il tipo di dato astratto \cod{Contenitore} mediante uso del costrutto \cod{class} del linguaggio. Un Contenitore pu� contenere istanze del tipo Oggetto, definito come segue:

\begin{codequote}
  const int NMAX = 50;
  typedef char Nome[NMAX];
  typedef int Peso;

  struct Oggetto {
    Nome n;
    Peso p;
  };
\end{codequote}

Inoltre, ogni contenitore pu� ospitare oggetti fino al raggiungimento di un peso complessivo massimo, oltre il quale nessun altro oggetto pu� essere ospitato.

Di seguito � riportata la specifica dei metodi pubblici da implementare per la classe \cod{Contenitore}.

\begin{methodslist}

\method{Contenitore}{Peso}{\emptyset}{
Costruttore con parametro di tipo \cod{Peso}. Il parametro indica il peso massimo raggiungibile dalla totalit� degli oggetti presenti nel contenitore.
}

\method{\~{}Contenitore}{\emptyset}{\emptyset}{
Distruttore.
}

\method{Inserisci}{Nome,Peso}{bool}{
Inserisce nel contenitore un oggetto avente il nome e il peso specificato. Il metodo restituisce \cod{true} se l'inserimento va a buon fine, cio� se il peso dell'elemento da inserire non eccede la capacit� residua del contenitore, \cod{false} altrimenti.
}

\method{Svuota}{\emptyset}{\emptyset}{
Svuota il contenitori di tutti gli oggetti presenti in esso.
}

\method{PesoComplessivo}{\emptyset}{Peso}{
Restituisce il peso complessivo raggiunto dal contenitore.
}

\method{PesoResiduo}{\emptyset}{Peso}{
Restituisce il peso residuo per il raggiungimento della capacit� massima del contenitore.
}

\method{NumElem}{\emptyset}{unsigned int}{
Restituisce il numero di oggetti presenti nel contenitore.
}

\method{Stampa}{\emptyset}{\emptyset}{
Stampa le coppie (Nome, Peso) di tutti gli oggetti presenti nel contenitore.
}

\end{methodslist}

L'unico metodo (pubblico, privato o protetto) della classe \cod{Contenitore} che pu� utilizzare lo standard-output (\cod{cout}) � il metodo \cod{Stampa()}. Gli altri metodi dovranno restituire l'esito delle operazioni eseguite mediante gli opportuni parametri di passaggio riportati nelle specifiche.