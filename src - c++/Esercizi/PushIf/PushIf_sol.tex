\thesolution{Push If}
Nella parte privata della classe sono dichiarati i seguenti membri:

\begin{codequote}
  class Pila {
  private:
    ...
    const unsigned int _maxpush;
    unsigned int _currpush;
    void _Push(const TElem& e);
    ...
  };  
\end{codequote}

La variabile membro \cod{\_maxpush} tiene memoria di qual � il numero di inserimenti massimi consecutivi ammessi; il suo valore � inizializzato dal costruttore al valore del parametro di ingresso e mai pi� variato durante il ciclo di vita delle istanze della classe. La variabile membro \cod{\_currpush} tiene memoria del numero di inserimenti consecutivi correntemente effettuati. Ogni chiamata al metodo \cod{Push()} deve verificare che questo parametro non ecceda il valore massimo consentito. Il metodo privato \cod{\_Push()} � implementato come una classica \cod{Push()}.

Di seguito si riporta l'implementazione dei metodi richiesti dalla traccia.

\inputprogram{Esercizi/PushIf/Pila.cpp}