\exercise{Push Greater}
Si realizzi in linguaggio C++ il tipo di dato astratto \cod{Pila} mediante uso del costrutto \cod{class} del linguaggio e ricorrendo ad un'implementazione dinamica. Il tipo \cod{TElem} degli elementi della pila sia il tipo \cod{int}.

Di seguito � riportata la specifica dei metodi pubblici da implementare per la classe \cod{Pila}.

\begin{methodslist}

\method{Pila}{\emptyset}{\emptyset} {
Costruttore senza parametri.
}

\method{\~{}Pila}{\emptyset}{\emptyset} {
Distruttore.
}

\method{Push}{TElem}{\emptyset} {
Aggiunge sulla pila l'elemento specificato.
}

\method{PushGreater}{TElem}{bool} {
Aggiunge sulla pila l'elemento specificato esclusivamente se esso � maggiore dell'elemento di testa corrente. Nel caso in cui la pila sia vuota l'aggiunta � sempre eseguita. Restituisce \cod{true} oppure \cod{false} a seconda che l'aggiunta sia stata eseguita oppure no.
}

\method{Top}{\emptyset}{TElem} {
Restituisce l'elemento di testa corrente della pila (ma non lo estrae). In caso di pila vuota il comportamento di questo metodo � non specificato.
}

\method{Pop}{\emptyset}{TElem} {
Estrae e restituisce l'elemento di testa corrente della pila. In caso di pila vuota il comportamento di questo metodo � non specificato.
}

\method{Svuota}{\emptyset}{\emptyset} {
Svuota la pila.
}

\method{Count}{\emptyset}{unsigned int} {
Restituisce il numero di elementi presenti nella pila.
}

\method{Empty}{\emptyset}{bool} {
Predicato vero se la pila � vuota, falso altrimenti.
}
\end{methodslist}

Si realizzi una funzione \cod{main()} che permetta di effettuare il collaudo della struttura dati realizzata.

Nessuno dei metodi della classe pu� utilizzare operazioni che coinvolgono gli stream di input ed output (\cod{cin} e \cod{cout}). La scrittura e la lettura su stream sono concesse esclusivamente all'interno del programma \cod{main()}.