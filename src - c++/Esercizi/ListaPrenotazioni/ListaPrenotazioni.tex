\exercise{Lista Prenotazioni}
Si realizzi in linguaggio C++ il tipo di dato astratto \cod{ListaPrenotazioni} mediante uso del costrutto \cod{class} del linguaggio. La lista deve memorizzare le prenotazioni di studenti ad un generico evento (uno ed uno solo). Gli elementi della lista siano del tipo \cod{Prenotazione} specificato di seguito:

\begin{codequote}
  typedef int Matricola;
  typedef char Nome[30];

  struct Prenotazione {
    Matricola mat;
    Nome nom;
  };
\end{codequote}

I metodi da implementare per la classe \cod{ListaPrenotazioni} siano conformi alla seguente interfaccia.

\begin{methodslist}

\method{ListaPrenotazioni}{int}{\emptyset} {
Costruttore con parametro intero. Il parametro indica il numero massimo di posti disponibili per l'evento, oltre i quali non deve essere possibile inserire ulteriori prenotazioni.
}

\method{\~{}ListaPrenotazioni}{\emptyset}{\emptyset} {
Distruttore.
}

\method{Prenota}{Matricola,Nome}{bool} {
Se nella lista non � gi� presente alcuna altra prenotazione con lo stesso numero di matricola e se ci sono posti disponibili, inserisce una nuova prenotazione in coda alla lista. Il metodo restituisce l'esito dell'operazione.
}

\method{EliminaPrenotazione}{Matricola}{boolt} {
Elimina dalla lista la prenotazione relativa al campo matricola specificato (se presente). Il metodo restituisce \cod{true} se � stato eliminato un elemento, \cod{false} altrimenti.
}

\method{GetPostiDisponibili}{\emptyset}{int} {
Restituisce il numero di posti ancora disponibili.
}

\method{EsistePrenotazione}{Matricola}{bool} {
Restituisce \cod{true} se esiste la prenotazione relativa al numero di matricola specificato, \cod{false} altrimenti.
}

\method{Svuota}{\emptyset}{\emptyset} {
Svuota la lista.
}

\method{Stampa}{\emptyset}{\emptyset} {
Stampa il contenuto dell'intera lista nel formato seguente: \cod{Mat\-ri\-co\-la1: Nome1, Matricola2: Nome2, Matricola3: Nome3, ...}
}

\end{methodslist}

L'unico metodo della classe \cod{ListaPrenotazioni} che pu� utilizzare lo standard-output (\cod{cout}) � il metodo \cod{Stampa()}. Gli altri metodi (pubblici, privati o protetti) non possono fare uso degli stream di I/O.
 
Si realizzi una funzione \cod{main()} che permetta di effettuare il collaudo della struttura dati realizzata.