\exercise{Lista Della Spesa}
Si realizzi in linguaggio C++ il tipo di dato astratto \cod{ListaDellaSpesa} mediante uso del costrutto \cod{class} del linguaggio e ricorrendo ad un'implementazione dinamica. Gli elementi della lista siano del tipo \cod{Articolo} specificato di seguito:

\begin{codequote}
  typedef char Nome[20];
  typedef float Quantita;

  struct Articolo {
    Nome n;
    Quantita q;
  };
\end{codequote}

Di seguito si riporta la specifica dei metodi da implementare.

\begin{methodslist}

\method{ListaDellaSpesa}{\emptyset}{\emptyset} {
Costruttore.
}

\method{\~{}ListaDellaSpesa}{\emptyset}{\emptyset} {
Distruttore.
}

\method{Aggiungi}{Nome,Quantita}{Quantita} {
Se nella lista non � gi� presente alcun altro elemento con lo stesso nome, inserisce l'elemento specificato (nella quantit� specificata) in coda alla lista. Nel caso in cui invece l'elemento fosse gi� presente nella lista, vi aggiunge la quantit� specificata.

Il metodo restituisce la quantit� con cui l'articolo specificato � presente nella lista in seguito all'aggiunta.
}

\method{Elimina}{Nome}{bool} {
Elimina dalla lista l'elemento avente il nome specificato (se presente).
Il metodo restituisce \cod{true} se � stato cancellato un elemento, \cod{false} altrimenti.
}

\method{GetQuantita}{Nome}{Quantita} {
Restituisce la quantit� dell'elemento presente nella lista ed avente il nome specificato. Se l'elemento non � presente restituisce zero.
}

\method{Svuota}{\emptyset}{\emptyset} {
Svuota la lista.
}

\method{Stampa}{\emptyset}{\emptyset} {
Stampa il contenuto dell'intera lista nel formato \cod{Nome: Quantit�, Nome: Quantit�, \ldots}
}

\end{methodslist}

L'unico metodo della classe \cod{Lista\-Della\-Spesa} che pu� stampare sullo standard\--output (\cod{cout}) � il metodo \cod{Stampa()}. Gli altri metodi (pubblici, privati o protetti) non possono fare uso delle funzionalit� di stampa.

Si realizzi una funzione \cod{main()} che permetta di effettuare il collaudo della struttura dati realizzata.