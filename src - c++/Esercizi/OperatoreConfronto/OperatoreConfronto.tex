\exercise{Operatore di Confronto}
Dotare la classe \cod{AlberoBinario} (\cfr{Ex:Albero Binario}) dell'operatore membro di confronto. Tale operatore viene invocato in seguito alla valutazione della seguente espressione:
\begin{codequote}
  a1 == a2;
\end{codequote}  
(ad esempio in un costrutto \cod{if}) dove \cod{a1} ed \cod{a2} sono due istanze della classe \cod{AlberoBinario}. In questo caso viene invocato l'operatore \cod{operator==()} sull'oggetto \cod{a1}, mentre \cod{a2}, parametro attuale, viene passato per riferimento prendendo il posto del parametro formale dell'operatore.

Di seguito si riporta la specifica dell'operatore di confronto da realizzare.
  
\begin{methodslist}

\method{operator==}{AlberoBinario}{bool} {
� l'operatore di confronto tra alberi. Permette di valutare l'esatta uguaglianza di due alberi. Fornisce \cod{true} se esso stesso risulta essere perfettamente uguale all'albero in ingresso (anche strutturalmente), \cod{false} altrimenti.
}
\end{methodslist}